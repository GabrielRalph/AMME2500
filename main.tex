\documentclass{article}
\usepackage[utf8]{inputenc}
\usepackage{svg}
\usepackage{soul}
\usepackage{amsmath}
\usepackage{wrapfig}
\usepackage{amssymb}
\usepackage[framed,numbered,autolinebreaks,useliterate]{mcode}

\title{A1}
\author{Gabriel Ralph \\ 470205736}
\date{March 2020}

\begin{document}

\maketitle

\section{}
\subsection{}
\includesvg[width = \textwidth]{q1-i.svg}
$F_x = mg\mu$\\ \\
$ma_x = mg\mu$\\ \\
$a_x = g\mu$\\ \\
$a_x = \frac{dv}{dt}$\\ \\
let $v_{n1} - v_{n0} = dv$\\ \\
$g\mu dt = v_{n1} - v_{n0}$\\ \\
$v_{n1} = v_{n0} + g\mu$\\ \\
As the force due to friction will be acting in the opposite direction to motion let us assume $g = -9.81$ \\ \\
$v_{n1} = v_{n0} - 9.81\mu$\\ \\
further more $v = \frac{dx}{dt}$\\ \\
Let $dx = x_{n1} - x_{n0}$ \\ \\
$v_{n1} dt = x_{n1} - x_{n0}$\\ \\
$x_{n1} = x_{n0} + v_{n1}dt$\\ \\
\subsection{}
\lstinputlisting{simulateStop.m}
\subsection{}
\includesvg[width = \textwidth]{q1-iii.svg}
\subsection{}
\includesvg[width = \textwidth]{q1-iv.svg}
\section{}
\subsection{}
$a_t = 0.5e^t$\\ \\
$dv = 0.5e^t dt$\\ \\
$\int dv_t = \int 0.5e^t dt$\\ \\
$v_t = 0.5e^t + C_1$\\ \\
Using the base case $v_t = \theta = t = 0$\\ \\
$0 = 0.5 + C_1$\\ \\
$C_1 = -0.5$\\ \\
$\therefore v_t = 0.5(e^t -1)$ \\ \\
At $t = 3$ \\ \\
$v_t = 0.5(e^3 - 1)$\\ \\
\hl{$v = 9.5428 m/s$}\\ \\
$a_t = 0.5e^3 = 10.0428$\\ \\
But we must also consider the centripetal acceleration $a_c = \cfrac{v^2}{r}$ \\ \\
$a_c = \cfrac{9.5428}{100} = 0.9106$\\ \\
$a = \sqrt{a_c^2 + a_t^2}= \sqrt{0.9106^2 + 10.0428^2}$\\ \\
\hl{$a = 10.0840 m/s^2$}
\subsection{}
$v = \frac{ds}{dt}$\\ \\
$\int ds = \int 0.5(e^t - 1)dt$\\ \\
$s = 0.5(e^t - t + C_2)$\\ \\
Using the base case $\theta = s = t = 0$\\ \\
$0 = 1 - 0 + C_2$\\ \\
$C_2 = -1$\\ \\
$\therefore s = 0.5(e^t - t-1)$\\ \\
$s = \theta r$\\ \\
$\theta = \cfrac{e^t - t - 1}{200}$\\ \\
$\theta = \frac{\pi}{6}$\\ \\
Using matlab to solve $\frac{\pi}{6} = \cfrac{e^t - t - 1}{200}$ returns the following. \\ \\
$ t = \cfrac{100\pi}{3} - lambertw(-1, -e^{- \frac{100\pi}{3} - 1}) - 1 = 4.7043$ \\ \\
\lstinputlisting{A1Q2ii.m}
Substituting $t = 4.7043$ into the derived equations for $v_t$, $a_t$ and $a_c$ gives the following\\ \\
\hl{$v = 54.712 m/s$} \\ \\
\hl{$a = 62.805 m/s^2$} \\ \\
\section{}
\subsection{}
The tension in the cord at $\theta = 0$ is simply the force the child exerts due to gravity i.e.\\ \\
$\Sigma F_y = 0 = T - mg$\\ \\
$T = mg$ \\ \\
\hl{$T = 294.3N$}
\subsection{}
We know that for any given position the sum of kinetic energy and gravitational potential energy will remain constant, such that\\ \\
$mgh_1 + \frac{1}{2}mv_1^2 = mgh_2 + \frac{1}{2}mv_2^2$\\ \\
Considering the two positions\\ \\
1. The tyre is at $\theta_1 = 0$ and hence $h_1 = 0$ and $v_1 = 4m/s$\\ \\
2. The tyre is at some position $\theta_2$ where $h_2 = L(1 - cos(\theta_2))$ and the tyre has come to a momentary rest such that $v_2 = 0$\\ \\
$\frac{1}{2}mv_1^2 = mgh_2$\\ \\
$\frac{1}{2}v_1^2 = gL(1-cos(\theta))$\\ \\
$cos(\theta) = 1 - \cfrac{v_1^2}{2gL}$\\ \\
$\theta = cos^{-1}(0.7961)$\\ \\
\hl{$\theta = 0.65 (rad) = 37.24^{\circ}$}
\subsection{}
At the instance where $\theta = 37.24^{\circ}$ \\ \\
$Fw = mg$\\ \\
$Fw_c = Fwcos(\theta_2)$\\ \\
$\Sigma F_c = 0 = T - Fw_c$\\ \\
$T = mgcos(\theta_2)$\\ \\
\hl{$T = 234.3N$}

\end{document}
